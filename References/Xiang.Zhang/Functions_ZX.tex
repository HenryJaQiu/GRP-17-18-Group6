\documentclass{article}

\begin{document}

\section{Functions}
The software is a teaching software used to illustrate how the Sequential Monte Carl algorithm works by using visualization techniques. Therefore, all the functions are designed to lead students to understand the process of  algorithm and the affect of arguments. Furthermore, as a teaching demonstration tool, making users comprehend it quickly is more significant than finishing tough missions through complicated operations. The following part will declare the functions designed according to user requirements and user-friendly principle.
    \subsection{Introduction}
     Here is a brief introduction of how user can use the software. All functions will be explained in details in next part.

     After opening the software, the first step is "\textbf{choose an algorithm}" to select one algorithm from Sequential Monte Carl algorithms which have different approaches to deal with data.% 等算法具体内容
     Afterwards, Users can import files in the software or using exist sample data. The result will be showed as both image and data. Otherwise, users can choose to change the value of some argument and run it again. The software will show the comparison of these two performances which can assist user to understand the meaning of arguments directly.

    \subsection{Functions Details}
    \subsubsection{Common Functions}
   There are a few functions which softwares need commonly. Below is the common functions in the software.

      \begin{itemize}
        \item \textbf{Restart}

        When some error occurs or some setting changes, user can restart the software by clicking "Restart" button.
        \item \textbf{Exit}

        Exit the software.
        \item \textbf{Manual}

        Manual is the instruction of the software. The content contains the introduction of software ,information of version updating and operation guide. General problems can be solved by checking manual.
        \item \textbf{Document of File Format}

         Imported file will be extracted by uniform approach. The document provides users with correct format of file and required information that imported files need to contain.
        \item \textbf{About Us}

        Provide the personal and contact information of developers. If any problem occurs in use, user can contact developers to fix it.
     \end{itemize}


    \subsubsection{Specific Functions}
       \begin{itemize}
       \item\textbf{page 1}
        \begin{itemize}
          \item \textbf{select a specific algorithm}

          Sequential Monte Carl have amount of methods to process data in resampling procedure %or …
          Thus, a specific algorithm which its brief explanation will be displayed can be selected by user in the first step.  
      \end{itemize}
      \item\textbf{main page}
      \begin{itemize}
          \item \textbf{Import file}

          The software allows user to import correct-format file. The contents stored in the file can be extracted as data which offer to algorithm.
          \item \textbf{Preset sample data}

          Several sample data are preset in the software. This function is use to satisfy the requirement that user only need a simple presentation without specific data. The arguments of sample data can be modified as well.
          \item \textbf{Run algorithm}
          
          Algorithm allows to use the data extracted from import file or sample data. After running the algorithm, the result is showed by two parts: Images and data. In addition, only two results based on the same original data can be showed in the screen simultaneously.
          \begin{itemize}
            \item Images

            There will be two lines in the image. One is the truly line created by offered data, another is created according to the inference of Sequential Monte Carl algorithm.%可改
            \item Data

            The crucial data of Sequential Monte Carl algorithm will be showed in the scene.% need details
          \end{itemize}
          \item \textbf{Clear}

          Clear is to clean previous data and image showed in the scene instead of imported file. Thus, users can run the same file multiple times to observe the image and data.
          \item \textbf{Modify value of arguments}

          This function allows user to modify part of data which can affect the performance of result. New result will be produced at the same page. According to modify value of arguments, the influence of each argument in the algorithm can be virtualized directly.
          \item \textbf{Compare with images}
          
          To display the different performance caused by different data, two images can be exist. These two images can be showed separately or combined in one image.  
          \item \textbf{Export}
          
          All the results can be exported by users. Because of the two forms of contents, export will be divided into two parts.  
          \begin{itemize}
            \item Images
            
            Both single image and combined image can be saved in PNG format or xxx. 
            \item Data
            
            Data can be saved in xxx format.
          \end{itemize}
       \end{itemize}
     \end{itemize}
\end{document}
