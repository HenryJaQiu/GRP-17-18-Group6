\documentclass[11pt,oneside,a4paper]{article}
%define the title
\usepackage{graphicx}

\author{Group}
\title{Midtern_report}
\begin{document}
\section{Particle filter}\\
\subsection{The basis of Particle filter}
It has been well known that a large amount of realistic data analysis tasks consist of estimating unknown quantities
from some given observations. Indeed, if the data are modeled as a linear Gaussian state-space model, it is possible
to derive a desired analytical expression to compute the evolving sequence of posterior distributions by implementing
Kalman filter, besides, it's still reachable to develop an analytical solution for the data are modeled as a partially
observed, finite state-space Markov chain. However, real data can be very complicated, typically involving elements of
non-Gaussianity, high dimensionality and nonlinearity, which conditions usually preclude analytic solution.
To deal with these sophisticated problems, many of approximation schemes such as Grid-based filters and Gaussian sum
approximations have been proposed to solve these problems.  One of these proposals named Sequential Monte Carlo(SMC)
methods(is equivalent to Partical filter) are a set of simulation-based methods which can be used to compute the posterior distributions.
Compared to other schemes, SMC is more flexible, easy to implement, parallelisable and applicable in very general
settings.\\



\subsection{The implementation of particle filter}
Particle filter can be implemented to solve a large number of practical application. A simple localization problem
is one of the example. Localization is the process of determining the position of an object with various onboard
sensors from which the aircraft can obtain the knowledge of its velocity and its altitude. A simple model of the aircraft
motion is provided by a (discrete-time) integrator
\begin{equation}
x_{t+1} = x_t + v_t + w_t,\qquad w_t \sim \mathcal{N}(0,5)
\end{equation}
where $x_t$ denotes the postion and $w_t$ denotes process noise.
The velocity can only help in establishing a relative position, which
means that additional information is needed to solve the localization
problem. One solution is to make use of a map of terrain
elevation and downward-facing radar, measuring the distance between the aircraft
and the ground. The corrsponding measurement equation is
\begin{equation}
y_t = h(x_t) + e_t,\qquad e_t \sim \mathcal{N} (0,1)
\end{equation}
where $y_t$ denotes the distance over ground measured by the radar, $h()$ denotes
a look-up in the map encoding the terrain elevation and $e_t$ denote the measurement
noise. However, the measurement equation (2) is nonlinear and the function $h()$ is only
defined in discrete points(according to the resolution of the map). But it fits the
Particle filter(PF) perfectly, since it can deal with nonlinear functions and a varying
number of possible hypotheses.
\includegraphics[width=1\textwidth]{./localization.png}
The result of implementing the PF with N = 200 particles is shown in the above picture.
At time 1, the figute(upper left) illustrates the situation that the aircraft is flying at an altitude of
roughly 80 m at position $x_1$ = 22 m,where the terrain elevation is 20 m.
Now, compare these positions to the PDF $\hat{p}^N(x_1|y_1)$ provided by the PF after the first measurement has been received and processed(upper right plot).
As we can see from the upper right plot, the PF provides several possible dominating modes representing the estimated position.
One dominating mode is located in $x$ = 22, which is corresponding with the upper right plot. When more measurement and information have
been received and processed, PF can estimate the more accurate position. The principle illustrated in this example of using a PF to solve the
localization problem by combining the information from sensors and maps has been successfully used to solve many different localization problems, including
for example underwater vessels, ships, cars and people moving around in both indoor and outdoor environments.\\

\subsection{The algorithm of bootstrap Particle filters}

Here, we introduce the bootstrap particle filters which is the simple representation of Particle filters.
The nonlinear filtering problem amounts to computing the filtering PDFs $ \{ p(x_t|y_{1:t}) \}$ Sequentially in time.
One principle solution is implemented by Forward filtering
\begin{equation}

p(x_t | y_{1:t}) = \frac{p(y_t|x_t)p(x_t|y_{1:t-1})}{p(y_t|y_{1:t-1})},

\end{equation}

\begin{equation}
p(x_t | y_{1:t-1}) = \int p(x_t|x_{t-1})p(x_{t-1}|y_{1:t-1}) d_{x_{t-1}},
\end{equation}

As you can see from the equation (4), to solve the $ \{ p(x_t|y_{1:t}) \}$, the integral in (4) will need to be handled, which can be approximated using an importance sampler
targeting the filtering distribution at time $t-1$.

\end{document}
